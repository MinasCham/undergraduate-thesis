\pagestyle{plain}
\begin{center}
{\LARGE Περίληψη}\\[1cm]
\end{center}

Αντικείμενο της παρούσας διπλωματικής εργασίας είναι η μελέτη και η κατανόηση των πρωτεϊνικών αλληλεπιδράσεων (Protein Protein Interactions - PPIs ), καθώς και η δημιουργία μοντέλων μηχανικής μάθησης (machine learning models) για τον εντοπισμό των PPIs. 

\medskip
H αλληλεπίδραση μεταξύ πρωτεϊνών αποτελεί ένα αντικείμενο μεγάλης σημασίας για την κατανόηση της λειτουργίας των πρωτεϊνών και έναν από τους βασικούς στόχους της συστημικής βιολογίας (systems biology ) \cite{Rao2014}. Αφού πρώτα παρουσιαστεί το θεωρητικό βιολογικό υπόβαθρο της εργασίας, γίνεται αναφορά στους τρόπους εντοπισμού των εν λόγω αλληλεπιδράσεων, τόσο πειραματικά όσο και υπολογιστικά, ενώ το κεφάλαιο ολοκληρώνεται με την επιγραμματική αναφορά στις μεγαλύτερες βάσεις δεδομένων πρωτεϊνικών αλληλεπιδράσεων.

\medskip
Το δεύτερο μέρος της εργασίας αφορά την εισαγωγή στις βασικές έννοιες της μηχανικής μάθησης. Με τον όρο \textit{μηχανική μάθηση} αναφερόμαστε σε μεθόδους ανάλυσης και επεξεργασίας δεδομένων με σκοπό την αυτοματοποίηση μοντέλων μηχανών και υπολογιστικών συστημάτων. Δίνονται ορισμοί και αναπτύσσονται έννοιες όπως τα νευρωνικά δίκτυα που θα αναπτυχθούν λεπτομερώς και στην υλοποίηση της εργασίας.

\medskip
Έπειτα από την ολοκλήρωση του θεωρητικού πλαισίου της εργασίας, υλοποιείται η προσέγγιση της παρούσας διπλωματικής. Ξεκινάει με τον ορισμό του προβλήματος και την ανάλυση της εξαγωγής δεδομένων αλληλεπιδράσεων πρωτεϊνών. Έπειτα από τη δημιουργία του συνόλου δεδομένων, παρουσιάζονται τεχνικές \textit{συμπλήρωσης μητρώων} (\textit{matrix completion}), μαθηματικές μέθοδοι συμπλήρωσης μη-υπαρχόντων τιμών στο σύνολο των δεδομένων, με σκοπό την μείωση της διαστασιμότητας του προβλήματος και την βελτίωση της απόδοσης των αλγορίθμων μας. Παρουσιάζεται μια τεχνική παραγοντοποίησης μητρώου μέσω stochastic gradient descent (matrix factorization with SGD), καθώς και μια τεχνική τανηστικής αποδόμησης (tensor decomposition). Έπειτα από την επεξεργασία των δεδομένων ακολουθεί η ανάπτυξη νευρωνικών δικτύων για τον εντοπισμό αλληλεπιδράσεων. Παρουσιάζονται λεπτομερώς οι αρχιτεκτονικές που χρησιμοποιήθηκαν στα πλαίσα της εργασίας, ενώ συγκρίνεται η απόδοση των εν λόγω δομών με υπάρχοντα μοντέλα. Η εργασία ολοκληρώνεται με την παρουσίαση των μελλοντικών σκέψεων σχετικά με την εξέλιξη των μοντέλων και την βελτίωση των αποτελεσμάτων.

\medskip
\noindent{\large\textbf{Λέξεις-Κλειδιά:}} Πρωτεΐνες, αλληλεπιδράσεις πρωτεϊνών, συμπλήρωση μητρώου, παραγοντοποίηση μητρώου, τανυστική αποδόμηση, μηχανική μάθηση, νευρωνικά δίκτυα.

