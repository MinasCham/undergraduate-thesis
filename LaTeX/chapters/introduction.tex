%!TEX root = ../main.tex

\chapter*{Εισαγωγή}
\markboth{Εισαγωγη}{}
%\vspace{-1.3in}

Οι πρωτεΐνες αποτελούν μια κατηγορία μακρομορίων που ελέγχει όλες τις βιολογικές διαδικασίες μέσα σε ένα κύταρρο και είναι άρρηκτα συνδεδεμένες με την υγεία των ανθρώπων. Παρ' ότι αρκετές πραγματοποιούν τις διεργασίες τους ως ανεξάρτητες οντότητες, η πλειοψηφία των πρωτεϊνών αλληλεπιδρά με άλλες πρωτεΐνες για την ορθή βιολογική δραστηριότητα. Επομένως, για την καλύτερη κατανόηση της λειτουργίας των πρωτεϊνών και συνεπώς και των οργανισμών, είναι σημαντικό ζήτημα ο εντοπισμός και η μελέτη των αλληλεπιδράσεων μεταξύ των πρωτεϊνών (γνωστές και ως \textit{Protein Protein Interractions - PPI}). Αν και στη θεωρία φαίνεται ως μια απλή διαδικασία, πρακτικά αποτελεί ένα εξαιρετικά δύσκολο ζήτημα, διότι τα κύταρρα αντιδρούν σε μια πληθώρα ερεθισμάτων, καθιστώντας την έκφραση των πρωτεϊνικών διεργασιών μια δυναμική διαδικασία με πολλαπλές παραμέτρους. Παράλληλα, οι πρωτεΐνες που συμμετέχουν σε μια αλληλεπίδραση μπορεί να μην είναι πάντα ενεργοποιημένες ή εμφανείς (στην πραγματικότητα οι περισσότερες αλληλεπιδράσεις είναι προσωρινού χαρακτήρα και απαιτούν έναν αριθμό συνθηκών για να πραγματοποιηθούν). 

\medskip
Οι ίδιες οι αλληλεπιδράσεις αποτελούνται από εναν συνδυασμό υδροφοβικών δεσμών με δυνάμεις \textit{Van der Waals} και \textit{γέφυρες αλάτων} (\textit{salt bridges}) σε συγκεκριμένες περιοχές (\textit{binding domains}) σε κάθε πρωτεΐνη. Ο εντοπισμός τους γίνεται εργαστηριακά μέσω μιας σειράς βιοχημικών πειραμάτων, τόσο \textit{in vivo} οσο και \textit{in vitro} (π.χ. co-immunoprecipation, TAP tagging, X-ray crystallography, Yeast two-hybrid κ.α.). Ωστόσο, ο τεράστιος αριθμός πιθανών αλληλεπιδράσεων σε συνδυασμό με το μεγάλο κόστος των παραπάνω πειραμάτων καθιστά αναγκαία τη δημιουργία αυτοματοποιημένων προσεγγίσεων για τον εντοπισμό ή/και την πρόβλεψη αλληλεπιδράσεων. Οι προσεγγίσεις αυτές, εκτελεσμένες σε υπολογιστή ή μέσω υπολογιστικής προσωμοίωσης, ονομάζονται μέθοδοι \textit{in silico} και αποτελούν κομμάτι της βιοπληροφορικής (\textit{bioinformatics}), τομέα που αφορά την κατασκευή μεθόδων και εργαλείων λογισμικού για την κατανόηση και επεξεργασία βιολογικών δεδομένων. Η συγκεκριμένη εργασία ασχολείται με δομικές προσεγγίσεις in silico τεχνικών, οπου προβλέπεται αλληλεπίδραση μεταξύ δυο πρωτεϊνών με βάση τα δομικά τους χαρακτηριστικά. Ειδικότερα, ασχολείται με τη δημιουργία μοντέλων μηχανικής μάθησης για την πρόβλεψη αλληλεπιδράσεων μεταξύ πρωτεϊνών.

\medskip
Η αύξηση της υπολογιστικής ισχύος έχει οδηγήσει τα τελευταία χρόνια στην ραγδαία ανάπτυξη του αντικειμένου της μηχανικής μάθησης, με τα μοντέλα μηχανικής μάθησης να αποδίδουν εξαιρετικά σε πολλούς τομείς όπως η υγεία, η βιολογία, τα οικονομικά κ.α. Στον τομέα της βιοπληροφορικής, η μηχανική μάθηση εφαρμόζεται σε τομείς τομείς όπως η \textit{γονιδιωματική} (\textit{genomics}), η \textit{εξέλιξη} (\textit{evolution}) και η βιολογία συστημάτων (\textit{systems biology}). Πλεόν, ο μεγάλος αριθμός βιολογικών δεδομένων δεν αποτελεί ανασταλτικό παράγοντα, αλλά το ερώτημα μετατοπίζεται στην κατανόηση του τεράστιου όγκου πληροφορίας και στην αξιοποίησή του. Ακολουθίες, δισδιάστατες και τρισδιάστατες δομές, καθώς και δίκτυα αλληλεπιδράσεων αποτελούν μόνο μερικά από τα δεδομένα ενδιαφέροντος. Όσον αφορά την πρόβλεψη αλληλεπιδράσεων πρωτεϊνών, η μηχανική μάθηση φαίνεται να έχει εξαιρετικά αποτελέσματα (Support Vector Machines \cite{Bradford2005}, Random Forests \cite{Mengying2017}, Convolutional Neural Networks \cite{Xie2020}, Bayesian networks \cite{NEUVIRTH2004} etc ).

\medskip
Στην διπλωματική εργασία κατασκευάστηκαν νευρωνικά δίκτυα με σκοπό την πρόβλεψη αλληλεπιδράσεων μεταξύ πρωτεϊνών, ωστόσο ακολουθήθηκε μια νεα προσέγγιση όσον αφορά την επεξεργασία των δεδομένων. Ειδικότερα, εφαρμόστηκαν μέθοδοι συμπλήρωσης μητρώων (\textit{matrix completion}), οπου συμπληρώθηκαν οι ελλιπείς τιμές των δεδομένων με βάση τις υπόλοιπες παρατηρίσιμες τιμές. Οι τεχνικές αυτές αποτελούν μια μορφή συνεργατικού φιλτραρίσματος (\textit{collaborative filtering}), κατά την οποία επιχειρούμε να εξάγουμε πληροφορία ή μοτίβα για τη συμπλήρωση τιμών συμπεριλαμβάνοντας πληροφορίες από πολλαπλές εγγραφές \cite{Terveen2001}. Οι μέθοδοι που χρησιμοποιήθηκαν ήταν η \textit{παραγοντοποίηση μητρώου μέσω Stochastic Gradient Descent} (\textit{Matrix Factorization with SGD}) και η \textit{τανηστική αποδόμηση CP} ( \textit{CP Tensor Decomposition}), για τις οποίες γίνεται λεπτομερής ανάλυση στη συνέχεια.

\medskip
Επειτα από την επεξεργασία των δεδομένων, εκπαιδεύτηκαν μοντέλα νευρωνικών δικτύων, και συγκεκριμένα ένα πλήρως διασυνδεδεμένο νευρωνικό δίκτυο (\textit{fully connected neural network}) και ένα συνελικτικό νευρωνικό δίκτυο (\textit{convolutional neural network}), ενώ αξιολογήθηκε η απόδοση τους σε σύγκριση με αντίστοιχες δημοσιευμένες εργασίες. Οι μετρικές που χρησιμοποιήθηκαν για την αξιολόγηση των μοντέλων ήταν: Accuracy, Precision, Sensitivity, Specificity, F-score και Matthews Correlation Co\-efficient (MCC). 



